
%**************************************************************
% Acronimi
%**************************************************************
\renewcommand{\acronymname}{Acronimi e abbreviazioni}

\newacronym[description={\glslink{apig}{Application Program Interface}}]
    {api}{API}{Application Program Interface}

\newacronym[description={\glslink{cto}{Chief Technical Officer}}]
    {cto}{CTO}{Chief Technical Officer}

\newacronym[description={\glslink{cpo}{Chief Process Officer}}]
    {cpo}{CPO}{Chief Process Officer}

\newacronym[description={\glslink{ceo}{Chief Executive Officer}}]
    {ceo}{CEO}{Chief Executive Officer}

\newacronym[description={\glslink{html}{HyperText Markup Language}}]
    {html}{HTML}{HyperText Markup Language}

\newacronym[description={\glslink{rest}{Representational State Transfer}}]
    {rest}{REST}{Representational State Transfer}    
   
%**************************************************************
% Glossario
%**************************************************************
\renewcommand{\glossaryname}{Glossario}

\newglossaryentry{acceleratore}
{
    name=\glslink{acceleratore}{acceleratore},
    text=acceleratore,
    sort=acceleratore,
    description={Nell'ambito della startup l'acceleratore indica un'associazione che fornisce sostegno e assistenza all'azienda nella sua fase primordiale, ``accelerando'' il suo processo di sviluppo mettendo quest'ultima nell'ambiente giusto e a stretto contatto con il mondo esterno e la rete degli investitori. Normalmente queste associazioni organizzano un vero e proprio programma di accelerazione, fornendo un investimento iniziale all'azienda. In cambio solitamente l'acceleratore detiene una quota societaria.}
}

\newglossaryentry{start-up}
{
    name=\glslink{start-up}{start-up},
    text=start-up,
    sort=start-up,
    description={Identifica la fase iniziale per l'avvio di una nuova impresa, cioè quel periodo nel quale un'organizzazione cerca di rendere profittevole un'idea attraverso processi ripetibili e scalabili. Inizialmente il termine veniva usato unicamente per indicare la fase di avvio di aziende nel settore internet o tecnologie dell'informazione.}
}

\newglossaryentry{agile}
{
    name=\glslink{agile}{agile},
    text=agile,
    sort=agile,
    description={Nell'ingegneria del software si riferisce ad una metodologia di sviluppo del software contrapposto ai modelli tradizionali (ad esempio \textit{waterfall}), in cui si pone l'attenzione sul consegnare al cliente il prodotto in tempi brevi e con iterazioni veloci. Tale metodologia è rappresentata dal \textit{Manifesto Agile}.\footnote{http://agilemanifesto.org/}}
}

\newglossaryentry{scrum}
{
    name=\glslink{scrum}{scrum},
    text=scrum,
    sort=scrum,
    description={È un framework di sviluppo agile del software iterativo ed incrementale in cui vengono enfatizzati tutti gli aspetti di progetto legati a contesti in cui è difficile pianificare in anticipo e in cui generalmente i requisiti cambiano molto spesso. Vengono istanziati meccanismi di controllo empirico, in cui al principio di \textit{``command-and-control''} viene contrapposta una tecnica di gestione basata su rapidi \textbf{cicli di feedback}.}
}

\newglossaryentry{Angular.js}
{
    name=\glslink{Angular.js}{Angular.js},
    text=Angular.js,
    sort=Angular.js,
    description={Si tratta di un framework per lo sviluppo web mantenuto dalla Google creato nel 2009. Favorisce un approccio di tipo dichiarativo allo sviluppo \textit{client-side}, migliore per la creazione di interfacce utente, laddove l'approccio imperativo si presta maggiormente per la realizzazione della logica applicativa. Esso si ispira fortemente al design pattern MVC riducendo considerevolmente il codice necessario alla realizzazione di applicazioni HTML/javascript.}
}

\newglossaryentry{API}
{
    name=\glslink{API}{API},
    text=API,
    sort=API,
    description={Indica ogni insieme di procedure disponibili al programmatore, di solito raggruppate a formare un set di strumenti specifici per l’espletamento di un determinato compito all’interno di un certo programma. La finalità è ottenere un’astrazione, di solito tra l’hardware e il programmatore o tra software a basso e ad alto livello semplificando così il lavoro di programmazione.}
}

\newglossaryentry{Bootstrap}
{
    name=\glslink{Bootstrap}{Bootstrap},
    text=Bootstrap,
    sort=Bootstrap,
    description={È una raccolta di strumenti liberi per la creazione di siti e applicazioni per il web. Essa contiene modelli di progettazione basati su HTML e CSS, sia per la tipografia, che per le varie componenti dell'interfaccia, come moduli, bottoni e navigazione, e altri componenti dell'interfaccia, così come alcune estensioni opzionali di JavaScript.}
}

\newglossaryentry{brainstorming}
{
    name=\glslink{brainstorming}{brainstorming},
    text=brainstorming,
    sort=brainstorming,
    description={Tecnica eseguibile in gruppo volta alla raccolta di idee per la soluzione di un problema.}
}

\newglossaryentry{CEO}
{
    name=\glslink{CEO}{CEO},
    text=CEO,
    sort=CEO,
    description={Traducibile in italiano come \textit{amministratore delegato}, nel mondo delle startup è un componente del team che si occupa normalmente del lato \textit{business} dell'impresa, è colui che fornisce un'interfaccia tra l'azienda e il mondo esterno.}
}

\newglossaryentry{Cloud Service Provider}
{
    name=\glslink{Cloud Service Provider}{Cloud Service Provider},
    text=Cloud Service Provider,
    sort=Cloud Service Provider,
    description={Si tratta di una struttura o un'organizzazione che fornisce servizi cloud ai clienti, a fronte di un contratto di fornitura.}
}

\newglossaryentry{CPO}
{
    name=\glslink{CPO}{CPO},
    text=CPO,
    sort=CPO,
    description={È un responsabile di alto livello all'interno di una startup che si occupa della gestione e coordinazione dei processi aziendali.}
}

\newglossaryentry{commit}
{
    name=\glslink{commit}{commit},
    text=commit,
    sort=commit,
    description={Nell'ambito di un repository indica l'operazione con cui si genera una versione del proprio lavoro, aggiungendo un nodo al ramo di sviluppo.}
}

\newglossaryentry{CTO}
{
    name=\glslink{CTO}{CTO},
    text=CTO,
    sort=CTO,
    description={È il responsabile del comparto tecnico all'interno di una startup che si occupa della valutazione e selezione delle tecnologie che possono essere applicate al prodotto o ai servizi che l'azienda produce.}
}

\newglossaryentry{Git}
{
    name=\glslink{Git}{Git},
    text=Git,
    sort=Git,
    description={È un sistema software di controllo di versione distribuito, creato da Linus Torvalds nel 2005.}
}

\newglossaryentry{incubatore}
{
    name=\glslink{incubatore}{incubatore},
    text=incubatore,
    sort=incubatore,
    description={Il concetto di incubatore è molto simile a quello di \textit{acceleratore}, indicando più precisamente il ``luogo fisico'' all'interno del quale risiede e lavora la startup.}
}

\newglossaryentry{HTML5}
{
    name=\glslink{HTML5}{HTML5},
    text=HTML5,
    sort=HTML5,
    description={È un linguaggio di markup per la strutturazione delle pagine web, pubblicato come W3C Reccomendation nell'Ottobre del 2014. Le novità introdotte dall'HTML5 rispetto all'HTML4 sono finalizzate soprattutto a migliorare il disaccoppiamento fra struttura, definita dal markup, caratteristiche di resa, definite dalle direttive di stile, e contenuti di una pagina web, definiti dal testo vero e proprio. Inoltre l'HTML5 prevede il supporto per la memorizzazione locale di grosse quantità di dati scaricati dal web browser, per consentire l'utilizzo di applicazioni basate su web anche in assenza di collegamento a Internet.}
}

\newglossaryentry{merge}
{
    name=\glslink{merge}{merge},
    text=merge,
    sort=merge,
    description={Nell'ambito di un repository indica la procedura tramite la quale un \textit{branch} viene unito in un altro, generalente nel branch che lo ha creato.}
}

\newglossaryentry{MongoDB}
{
    name=\glslink{MongoDB}{MongoDB},
    text=MongoDB,
    sort=MongoDB,
    description={È un database non relazionale (NoSQL), orientato ai documenti. Classificato come un database di tipo NoSQL, MongoDB si allontana dalla struttura tradizionale basata su tabelle dei database relazionali in favore di documenti in stile JSON con schema dinamico (MongoDB chiama il formato BSON), rendendo l'integrazione di dati di alcuni tipi di applicazioni più facile e veloce.}
}

\newglossaryentry{Node.js}
{
    name=\glslink{Node.js}{Node.js},
    text=Node.js,
    sort=Node.js,
    description={È un framework \textit{event-driven} di utilizzo \textit{server-side} di Javascript}
}

\newglossaryentry{REST}
{
    name=\glslink{REST}{REST},
    text=REST,
    sort=REST,
    description={Riferisce ad un insieme di principi di architetture di rete, i quali delineano come le risorse sono definite e indirizzate. Il termine è spesso usato nel senso di descrivere ogni semplice interfaccia che trasmette dati su HTTP.}
}

\newglossaryentry{peer-to-peer}
{
    name=\glslink{peer-to-peer}{peer-to-peer},
    text=peer-to-peer,
    sort=peer-to-peer,
    description={È un'espressione che indica un'architettura logica di rete informatica in cui i nodi non sono gerarchizzati unicamente sotto forma di client o server fissi (clienti e serventi), ma sotto forma di nodi equivalenti o paritari (in inglese peer) che possono cioè fungere sia da cliente che da servente verso gli altri nodi terminali (host) della rete. Nell'ambito di CoffeeStrap indica che l'utente svolge sia il ruolo di ``insegnante'' che quello di ``studente'' per l'apprendimento linguistico.}
}

\newglossaryentry{sprint}
{
    name=\glslink{sprint}{sprint},
    text=sprint,
    sort=sprint,
    description={Nella metodologia scrum viene inteso come il lasso di tempo che intercorre tra uno scrum meeting e il successivo. In particolare è la misura della dimensione di un'iterazione sul prodotto.}
}
