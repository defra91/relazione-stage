% !TEX encoding = UTF-8
% !TEX TS-program = pdflatex
% !TEX root = ../tesi.tex
% !TEX spellcheck = it-IT

%**************************************************************
\chapter{Il progetto di stage}
\label{cap:descrizione-stage}
%**************************************************************

\section{Pianificazione del lavoro}

In questa sezione descriverò come il progetto è stato suddiviso nell'arco delle otto settimane e quali erano gli obiettivi alla fine di ciascun periodo.

\section{Norme, procedure e strumenti}

In questa sezione descriverò tutte le norme e le procedure con il quale ho organizzato il mio lavoro, quali processi ho attuato con le loro motivazioni e di quali strumenti mi sono avvalso per l'attuazione di tali processi.

\section{Periodo di formazione}

\subsection{Apprendimento della tecnologia}

In questa sezione descriverò come ho appreso la tecnologia di riferimento e attraverso quali processi, approfondendo nel dettaglio alcuni concetti indispensabili e di rilievo. Descriverò la differenza tra le mie conoscenze alla fine di questo periodo e all'inizio, fornendo evidenza di apprendimento.

\subsection{Integrazione con le tecnologie aziendali}

In questa sezione descriverò come, dopo uno studio della tecnologia legata al progetto di stage abbia integrato quest'ultima con le tecnologie già esistenti utilizzate all'interno dell'azienda.

\section{Progettazione}

\subsection{Progettazione del flusso utente}

In questa sezione descriverò la progettazione del flusso utente all'interno dell'applicazione Android, avvalendomi di diagrammi di attività e di flusso.

\subsection{Progettazione del layout}

In questa sezione descriverò la progettazione del layout delle diverse schermate dell'applicazione, giustificando le varie scelte secondo criteri di accessibilità e usabilità.

\subsection{Progettazione architetturale}

In questa sezione descriverò la progettazione dell'architettura software dell'applicazione, andando ad evidenziare la suddivisione in classi e pacchetti, le interazioni di quest'ultimi, le librerie esterne e i design pattern utilizzati; il tutto in relazione con le best-practise indicate dalla comunitò di sviluppatori Android.

\section{Sviluppo e codifica}

In questa sezione descriverò come ho realizzato il codice dell'applicazione aderendo alla progettazione attuata in precedenza.

\section{Rifinitura e testing}

In questa sezione descriverò come a fronte dello sviluppo e codifica dell'applicazione ho proceduto alla sua verifica, istanziando un sistema di bug tracking e testando l'applicazione con utenti reali.

\section{Validazione del prodotto e rilascio}

In questa sezione infine descriverò come a fronte dell'attività di rifinitura è stata istanziata un'attività di validazione mirata a verificare che il prodotto corrispondesse alle aspettative e soddisfasse gli obiettivi preposti. Inoltre descriverò l'attività di rilascio e di pubblicazione di una versione sul play store di Google.