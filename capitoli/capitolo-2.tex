% !TEX encoding = UTF-8
% !TEX TS-program = pdflatex
% !TEX root = ../tesi.tex
% !TEX spellcheck = it-IT

%**************************************************************
\chapter{Il progetto all'interno dell'azienda}
\label{cap:progetto-azienda}
%**************************************************************

\section{Presentazione del progetto}

In questa sezione fornirò un'introduzione del progetto in questione.

\subsection{Motivazioni}

In questa sezione descriverò le motivazioni che hanno portato CoffeeStrap ad ospitare uno stage all'interno dell'azienda, quali vantaggi esso avrebbe portato e quale valore aggiunto si avrebbe avuto alla fine.

\subsection{Aspettative}

In questa sezione descriverò cosa l'azienda CoffeeStrap si aspettava da me e dal prodotto all'inizio dello stage.

\subsection{Obiettivi di formazione}

In questa sezione descriverò la strategia assunta da CoffeeStrap all'inizio dello stage legato al personale, ovvero al voler aggiungere un componente, un ruolo all'interno del team di sviluppo, legato anche al fatto che essi avevano già lavorato con me in precedenza nel progetto di Ingegneria del Software e quindi erano consci di quali fossero le skill da me acquisite.

\section{Vincoli}

\subsection{Vincoli tecnologici}

In questa sezione descriverò quali sono stati i vincoli tecnologici imposti per la realizzazione del progetto, come ad esempio gli ambienti di sviluppo, i framework e le librerie utilizzate. In particolare approfondirò le motivazioni che hanno portato alla scelta di sviluppare nativamente in Android piuttosto che utilizzare framework di sviluppo quali Titanium o Ionic, che erano stati inizialmente considerati.

\subsection{Vincoli metodologici}

In questa sezione descriverò le metodologie di sviluppo imposte dall'azienda per lo sviluppo del progetto.

\subsection{Vincoli temporali}

In questa sezione descriverò quali sono stati i vincoli temporali per la realizzazione degli obiettivi preposti.

