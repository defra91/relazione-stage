% !TEX encoding = UTF-8
% !TEX TS-program = pdflatex
% !TEX root = ../tesi.tex
% !TEX spellcheck = it-IT

%**************************************************************
\chapter{Il progetto all'interno dell'azienda}
\label{cap:progetto-azienda}
%**************************************************************

\section{Presentazione del progetto}

Il progetto svolto da me durante l'attività di stage consiste nello sviluppo di un'applicazione mobile, sviluppata nativamente in Android, che replica il meccanismo e il flusso utente già esistente nel web integrandolo all'interno di questo tipo di piattaforma, molto differente rispetto con uno stack tecnologico diverso e con diverse considerazioni in termini di usabilità.

\subsection{Motivazioni}

Come già presentato nel capitolo 1, CoffeeStrap ha come primo e unico scopo lo sviluppo di una piattaforma peer-to-peer per l'appredimento di una lingua tramite interazioni tra utenti. Lo scopo principale in questa prima fase di sviluppo è quella di raccogliere un bacino utenti considerevole, in modo da poter capire su quale linea portare avanti l'idea e modellare quest'ultima in base al feedback e al tipo di iterazione dell'utente medio. Prima dell'inizio del mio stage CoffeeStrap disponeva solamente dell'applicazione web, la quale possedeva circa 2000 iscritti e mediamente 15 utenti attivi al giorno. 

Il progetto di stage che ho svolto nasce dunque dalla necessità di affiancare all'applicazione web l'applicazione mobile, in modo da poter fornire all'utente più possibilità di utilizzo del prodotto ed arrivare inoltre ad un considerevole incremento del bacino utenti. La considerazione pricipale a monte di ciò è il fatto che oggigiorno l'utente medio spende la maggior parte del proprio tempo sul proprio smartphone piuttosto che sul proprio pc, per tutti i vantaggi legati alla tecnologia mobile. Questo argomento è chiaramente banale ed è oramai un dato assodato. Qualsiasi start-up in questo ambito se vuole avere successo deve prima o dopo nel proprio ciclo di vita sviluppare su mobile. È di fatto una necessità imprescindibile se si vuole comptere in un certo tipo di mercato, soprattutto se si parla di un'applicazione che può essere considerata un \textit{social network}.

L'avere a disposizione l'applicazione a qualsiasi ora del giorno e in qualsiasi luogo permette all'utente di avere a disposizione il prodotto senza limitazioni e nel momento esatto in cui ne sente la necessità. Da questo ne deriva un aumento del numero di iterazioni e della quantità di utenti attivi giornalieri, dati di fondamentale importanza per raccogliere investimenti.

\subsection{Aspettative}

La mia preparazione e formazione in ambito di sviluppo mobile poteva ritenersi pari a zero all'inizio dell'attività di stage. Cionostante l'azienda era consapevole di quali fossero le mie capacità, avendo già collaborato con me e il mio team per il progetto di ingegneria del software dell'anno accademico 2013/2014. Pochi mesi prima in collaborazione con il team \textbf{steakholders} avevamo infatti sviluppato un framework per l'amministrazione di dati provenienti da MongoDB, il quale l'azienda tuttora utilizza quotidianamente come strumento di supporto.

A fronte di questo progetto l'azienda poneva in me delle buone aspettative circa l'apprendimento della tecnologia e l'integrazione all'interno del team. Inoltre i membri di CoffeeStrap, avendo frequentato anch'essi il mio stesso corso di studi, erano al corrente di quale tipo di formazione l'università di Padova offrisse agli studenti e in particolare in quale modo il corso di \textit{Ingegneria del Software} preparasse questi ultimi ad essere inseriti velocemente nel mondo del lavoro. Queste motivazioni hanno fatto sì che l'azienda decidesse di ospitare al loro interno un'attività di stage che potesse soddisfare gli obiettivi preposti. 

Considerando la durata dello stage e le tempistiche legate all'apprendimento della tecnologia, l'aspettativa da parte dell'azienda era quella di arrivare alla fine del periodo di stage con in mano un prototipo funzionante, pronto per essere inserito in una fase successiva di test più approfondito e di raffinamento, comunque non ancora sufficientemente maturo per essere inserito pubblicamente nel mercato. L'aspettativa era quella che in ambito informatico viene generalmente indicata come versione \textit{beta}.

\subsection{Obiettivi di formazione}

L'azienda, essendo di piccole dimensioni e in fase primordiale, ha come prima necessità, oltre allo sviluppo e diffusione del prodotto, la formazione di un team qualificato con all'interno elementi di natura tecnica. A fronte di questa necessità ospitare un'attività di stage permette loro di poter integrare personale giovane, con una buona formazione e dotati di forte motivazione e stimoli, oltre che una buona elasticità mentale. 

Tutto ciò permette all'azienda di poter formare

\section{Vincoli}

\subsection{Vincoli tecnologici}



\subsection{Vincoli metodologici}

CoffeeStrap ha ritenuto importante il mio inserimento all'interno del team e l'apprendimento da parte mia delle loro metodologie di lavoro, pertanto fin dalla prima settimana sono stato inserito all'interno dei meccanismi di sviluppo agile, ho partecipato attivamente agli scrum meeting e agli stand-up meeting e ho creato e svolto le mie stories con gli stessi procedimenti attuati dagli altri membri. 

Oltre a questo c'è stato un monitoraggio costante da parte del \textit{tutor} aziendale sui processi e le metodologie da me instanziati per l'apprendimento e lo sviluppo della tecnologia, che hanno fatto sì che prendessi gradualmente confidenza con il \textit{modus operandi} adottato dall'azienda.

\subsection{Vincoli temporali}

Lo stage viene distribuito in 8 settimane di calendario, per una durata complessiva di 300-320 ore lavorative. Ogni settimana sono previste 40 ore, considerando una giornata lavorativa composta di 8 ore. Lo stage può dunque considerarsi un'attività a tempo pieno.

Nonostante questo vincolo temporale imposto le modalità operative non prevedevano un conteggio delle ore lavorative, ma erano basate sullo svolgimento delle \textit{stories}, descritte nel capitolo 1. A fronte di ciò posso affermare che quanto realizzato a livello di ore lavorative differisce in eccesso rispetto a quanto pianificato, in quanto la completa libertà di gestire il proprio tempo ha fatto sì che molto spesso le mie giornate lavorative si prolungassero oltre le 8 ore e che il sabato e la domenica (giorni non considerati in un calendario lavorativo) venissero frequentemente spesi nello sviluppo dell'applicazione.

In nessun modo CoffeeStrap ha imposto vincoli temporali superiori a quanto pattuito. Queste scelte da me effettuate sono state puramente personali, derivate dalla forte motivazione e sollecitazione date dal progetto, di per sè molto stimolante, e dalla crescente voglia di imparare.

